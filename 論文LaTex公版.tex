\documentclass{ctexart}

% 引入必要的包
\usepackage{tikz}
\usepackage{qtree} % 引入 qtree 套件
\usepackage{url} % 用於支持 \url 命令
\usepackage{lipsum} % 虛擬文本示例
\usepackage{setspace}
\usepackage{booktabs}
\usepackage{float}
\usepackage{colortbl}
\usepackage{xcolor}
\usepackage{indentfirst} % 使首段也縮進
\usepackage[a4paper, top=3cm, bottom=3cm, left=3cm, right=3cm]{geometry}
\usepackage{titlesec} % 添加 titlesec 宏包以設置章節標題格式
\usepackage{amsmath} % 用於數學公式
\usepackage{graphicx} % 用於插入圖片
\usepackage{fancyhdr} % 用於頁眉和頁腳
\usepackage{longtable}
\usepackage{array}
\usepackage{caption}
\usepackage{fontspec}
\usepackage{tabularx}
\usepackage{subcaption}
\usepackage{etoolbox} % 用於調整 \paragraph 標題格式
\usepackage{titletoc} % 用於目錄中添加 \paragraph
\usepackage{tocloft}
\usepackage{hyperref} % 最後加載 hyperref 包以生成目錄中的超鏈接

% 設置系統中已安裝的中文字體
\setCJKmainfont{STKaiti} % 標楷體
\setmainfont{Times New Roman}

% 使用 ctex 宏包提供的接口重新定義參考文獻標題
\ctexset{
    bibname = {參考文獻},
}

% 自定義章節標題格式
\titleformat{\section}[block]{\centering\fontsize{20}{24}\selectfont\bfseries}{第\thesection 章}{1em}{}
\titleformat{\subsection}[block]{\fontsize{16}{20}\selectfont\bfseries}{\thesubsection.}{1em}{}
\titleformat{\subsubsection}[block]{\fontsize{14}{18}\selectfont\bfseries}{\thesubsubsection.}{1em}{}
\titleformat{\paragraph}[block]{\normalfont\normalsize\bfseries}{\theparagraph}{1em}{}[]

% 重新定義章節計數器,確保章節編號是阿拉伯數字形式
\renewcommand{\thesection}{\arabic{section}}
\renewcommand{\thesubsection}{\thesection.\arabic{subsection}}
\renewcommand{\thesubsubsection}{\thesubsection.\arabic{subsubsection}}
\renewcommand{\theparagraph}{\thesubsubsection.\arabic{paragraph}}

% 設置頁碼格式和位置
\fancypagestyle{plain}{
  \fancyhf{}
  \fancyfoot[C]{\thepage} % 頁碼居中
  \renewcommand{\headrulewidth}{0pt}
  \renewcommand{\footrulewidth}{0pt}
}

\pagestyle{plain} % 設置頁碼樣式

% 設置圖以及表標題格式
\captionsetup[figure]{name=Figure}
\captionsetup[table]{name=Table}

% 修改圖表目次項的顯示格式
\renewcommand{\cftfigpresnum}{\figurename~}
\renewcommand{\cftfigaftersnum}{\hspace{1em}}
\renewcommand{\cfttabpresnum}{\tablename~}
\renewcommand{\cfttabaftersnum}{\hspace{1em}}
\renewcommand{\cftfigleader}{\cftdotfill{\cftdotsep}}
\renewcommand{\cfttableader}{\cftdotfill{\cftdotsep}}

% 調整目次中圖表標題的字體大小
\renewcommand{\cftfigfont}{\Large}
\renewcommand{\cfttabfont}{\Large}

% 修改圖表目次項的顯示格式
\titlecontents{figure} % 修改圖表目錄中的圖
  [0pt] % 左邊距
  {\addvspace{1em}\Large} % 前置間距並設置字體大小
  {\Large\bfseries Figure~\thecontentslabel\enspace} % 標號格式
  {} % 無標題格式
  {\bfseries\cftdotfill{\cftdotsep}\contentspage} % 標號後置格式,包含虛線
  
\titlecontents{table} % 修改圖表目錄中的表
  [0pt] % 左邊距
 {\addvspace{1em}\Large} % 前置間距並設置字體大小
  {\Large\bfseries Table~\thecontentslabel\enspace} % 標號格式
  {} % 無標題格式
  {\bfseries\cftdotfill{\cftdotsep}\contentspage} % 標號後置格式,包含虛線

% 確保段落顯示編號
\setcounter{secnumdepth}{4}
% 使用 etoolbox 宏包重新定義 paragraph 並設置字體大小和間距
\makeatletter
\patchcmd{\paragraph}{\normalfont\normalsize\bfseries}{\normalfont\normalsize\bfseries\fontsize{12}{16}\selectfont}{}{}
\patchcmd{\paragraph}{\theparagraph}{\theparagraph}{}{}
\titlespacing*{\paragraph}{0pt}{1.5ex plus 1ex minus .2ex}{1em}
\makeatother

    
% 調整目次項的間距
\setlength{\cftfignumwidth}{4em} % 調整圖目次項數字的寬度
\setlength{\cfttabnumwidth}{4em} % 調整表目次項數字的寬度
\setlength{\cftbeforesecskip}{1em}       % 調整章節標題前的間距
\setlength{\cftbeforesubsecskip}{1em}    % 調整小節標題前的間距
\setlength{\cftbeforesubsubsecskip}{1em} % 調整次小節標題前的間距


% 設置摘要頁碼格式
\pagenumbering{roman} % 使用小寫羅馬數字編號

% 目錄設置
\renewcommand{\cfttoctitlefont}{\Large\bfseries\CJKfamily{bkai}\hfill}
\renewcommand{\cftaftertoctitle}{\hfill} % 目次標題居中
\renewcommand{\cftloftitlefont}{\Large\bfseries\CJKfamily{bkai}\hfill}
\renewcommand{\cftafterloftitle}{\hfill} % 表目次標題居中
\renewcommand{\cftlottitlefont}{\Large\bfseries\CJKfamily{bkai}\hfill}
\renewcommand{\cftafterlottitle}{\hfill} % 圖目次標題居中
\setlength{\cftaftertoctitleskip}{10pt}

% 調整目錄項目的字體大小
\renewcommand{\cftsecfont}{\Large\CJKfamily{bkai}} % 章節標題
\renewcommand{\cftsubsecfont}{\Large\CJKfamily{bkai}} % 小節標題
\renewcommand{\cftsubsubsecfont}{\Large\CJKfamily{bkai}} % 小小節標題
\renewcommand{\cftparafont}{\Large\CJKfamily{bkai}} % 其他標題
\renewcommand{\cftbeforesecskip}{10pt} % 章節標題之前的間距
\renewcommand{\cftbeforesubsecskip}{8pt} % 小節標題之前的間距

\usepackage{geometry}
\geometry{
    textheight=20.5cm,
    top=3cm,
    bottom=3cm,
    footskip=1.5cm,
    marginparwidth=2.5cm,
    left=3cm,
    right=3cm
}

\begin{document}

\begin{titlepage}
    \centering
    {\fontsize{20}{24}\selectfont 國立中興大學應用數學系\\[0.5cm] 碩士論文 \par}
    \vspace{4.8cm} % 調整論文題目位置,將其與「國立中興大學OO系」相隔6公分
    {\fontsize{20}{24}\selectfont 這裡打論文中文題目\\(以K公司為例)\par}
    {\fontsize{20}{24}\selectfont 這裡打論文英文題目\par}
     \vspace{6.2cm}
    {\fontsize{20}{24}\selectfont \textbf{指導教授:}\setmainfont{Kaiti TC} OOO \textit{OOO} \par}  
     \vspace{0.5cm} % 調整行距
    {\fontsize{20}{24}\selectfont \textbf{研究生:}\setmainfont{Kaiti TC} OOO \textit{OOO} \par}
    \vfill
    {\fontsize{20}{24}\selectfont 中華民國OOO年O月 \par}
\end{titlepage}

\onehalfspacing % 1.5倍行距

% 致謝頁
\clearpage
\thispagestyle{empty} % 清空當前頁面的頁眉和頁腳
\begin{center}
    \Large \textbf{誌謝辭}
\end{center}
\vspace{0.5cm}
\noindent
\par
這裡打致謝辭內容
\newpage

% 中文摘要部分
\begin{center}
{\fontsize{20}{24}\selectfont 摘要} %設置標題
\setcounter{page}{1} % 從第1頁開始
\end{center}
\vspace{0.5cm} % 摘要與標題距离为0.5cm
\noindent % 段落不縮進
\par
這裡打摘要
% 添加中文摘要的關鍵詞
\vspace{1cm} % 设置与关键词之间的垂直距离为1cm
\noindent % 段落不缩进
\textbf{關鍵字:}XXX、OOO、...
% 摘要页码置中
\thispagestyle{plain} % 设置当前页页码格式为默认样式,没有页眉页脚但有页码
\addcontentsline{toc}{section}{摘要} % 添加摘要到目錄
\newpage

% 英文摘要部分
\begin{center}
    {\fontsize{20}{24}\selectfont Abstract} % 设置摘要标题的字体大小和样式
\end{center}
\vspace{0.5cm} % 设置与摘要标题的距离为0.5cm
\noindent % 段落不缩进
\par
這裡打英文拆要
% 添加英文摘要的關鍵詞
\vspace{1cm} % 设置与关键词之间的垂直距离为1cm
\noindent % 段落不缩进
\textbf{Keywords:}英文摘要關鍵字
% 英文摘要页码置中
\thispagestyle{plain} % 设置当前页页码格式为默认样式,没有页眉页脚但有页码
\addcontentsline{toc}{section}{Abstract} % 添加Abstract到目錄
\newpage


% 目录页
\renewcommand{\contentsname}{目次}
\tableofcontents
\addcontentsline{toc}{section}{目次} % 添加目次到目錄
\newpage

% 表目录页
\renewcommand{\listtablename}{表目次}
\listoftables
\addcontentsline{toc}{section}{表目次} % 添加表目次到目錄
\newpage

% 图目录页
\renewcommand{\listfigurename}{圖目次}
\listoffigures
\addcontentsline{toc}{section}{圖目次} % 添加圖目次到目錄
\newpage

% 修改图目次项的显示格式
\makeatletter
\renewcommand{\l@figure}[2]{%
  \noindent\makebox[2.5em][l]{\figurename\  #1:} #2\par}
\makeatother

% 调整页码格式为阿拉伯数字,并从 1 开始编号
\pagenumbering{arabic}
\setcounter{page}{1}

\section{緒論}

\subsection{研究背景}
\subsubsection{XXX} 
 

\section{文獻探討}

\subsection{XXX}
\subsubsection{XXX} 


\section{研究方法}
\subsection{XXX}
\subsubsection{XXX}

\section{實驗結果}
\subsection{XXX}
\subsubsection{XXX}

\section{結論與限制}
\subsection{XXX}
\subsubsection{XXX}

\clearpage % 插入新頁

% 插入參考文獻部分
\begin{thebibliography}{99}
\bibitem{sap}
現代化供應鏈的需求預測思愛普 SAP。取自 \url{https://www.sap.com/taiwan/products/scm/integrated-business-planning/what-is-supply-chain-planning/demand-forecasting.html}
\bibitem{zhangjian2022}張簡復中(2022), 供應鏈管理 (第三版), 新文京開發出版股份有限公司,新北市.
\bibitem{breiman2001random} Breiman, L. (2001). Random forests. Machine learning, 45, 5-32.

\end{thebibliography}
\end{document}
